\chapter{Conclusions}
\label{chap:conclusions}

    Hardware design is an important and \emph{complex} activity, with strict requirements.
    Furthermore, its relevance is growing with more need for hardware acceleration,
    pushing programmers and other professionals to think about hardware without having
    the traditional engineering background.

    Functional programming has for a long time allowed for programming in a more
    abstract fashion, farther from details of any specific machine.
    Also, very frequently programming a system using a purely functional programming language
    makes that system easier to analyze and for properties of its behaviour to be stated and proven.

    Given these advantages when developing software, it should come as no surprise that
    a long-standing line of research is concerned with applying the ideas of functional
    programming to hardware design.
    Functional hardware description languages, and later \emph{domain-specific languages}
    for hardware \emph{embedded} in general-purpose functional languages were developed,
    and several studies have shown that, indeed, some of the advantages brought to software development
    by adopting the functional paradigm could also benefit the world of hardware.

    More recently, programming languages with \emph{dependent type systems} were developed,
    creating the paradigm of \acl{DTP}.
    Many researchers regard \ac{DTP} as being the "next step up" in terms of abstraction,
    when compared to functional programming.
    Nothing more reasonable then than to investigate whether even more benefits can be brought to
    the field of hardware design by \ac{DTP},
    building upon the advances of functional hardware description languages.

    Inspired by the ideas in Coquet~\cite{coquet2011}, and by the strenghts and weaknesses of other
    hardware \acp{DSL} studied in a previous project, we aimed in this thesis to develop
    a hardware \ac{DSL} – Π-Ware – \emph{embedded in a dependently-typed language} and using
    as much as possible of the cutting-edge developments in \ac{DTP}.
    The development of Π-Ware had an exploratory character: based on the identified limitations of
    other libraries

    %% TODO: before the sections, "tie the knots" and give a general conclusion

    \section{Future work}
    \label{sec:future-work}

        \subsection{Current limitations of Π-Ware}
        \label{subsec:current-limitations}
            \begin{itemize}
                \item Synthesizable instances are not proven bijections?
                \item Big circuits consume too much time and memory to typecheck
                    \subitem Who to blame?
                    \subitem Instance search
                    \subitem Combinational proofs
                \item Circuit interfaces are described by \emph{just a size}
                    \subitem Simple, but generates ugly VHDL
                    \subitem All circuits have one big vector input and one output
                    \subitem Could we make the interface indices more \emph{structured}?
            \end{itemize}

        \subsection{General future research directions}
        \label{subsec:future-research}
            \begin{itemize}
                \item More broadly, considering \emph{Π-Ware}, what would be the ``next steps''.
                \item Map arbitrary Agda types to words (via sum-of-products) in a GP way.
                \item Prove more properties of metatheory using Agda
                    \subitem Identities of circuit constructors, associativity, other algebraic properties
                    \subitem State law of μFP
            \end{itemize}
