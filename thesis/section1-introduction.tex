\section{Introduction}
\label{sec:intro}
    Several factors have been causing an increasing demand for hardware acceleration of algorithms,
    and there is also pressure to reduce the duration and cost of the circuit design process.
    These trends collide with the techniques and tooling used in this activity, which are,
    compared to the ones observed in software development, primitive.

    % Hardware is harder :)  Stronger correctness requirements, verification, validation

    One of the (long-running) trends to increase productivity in the software industry
    is the usage of functional programming.
    Advocates of functional programming claim productivity gains of up to 10 times to
    programmers coming from an imperative environment.
    Another advantage usually attributed to functional programming is an increased capacity to
    \emph{reason} about your programs, specially to perform what is called \emph{equational reasoning}.
    These claims have been confirmed in practice (and still keep getting more confirmations),
    and furthermore, functional techniques and constructs keep "penetrating" imperative languages
    with each new release. % mention C++11, SWIFT, etc.

    In a certain way, we can compare the level of abstraction in which hardware design and verification
    is performed nowadays with software development in its early age.
    Of course there are inherent and fundamental differences between the two activities, but
    this comparison leads us to ask whether current research ideas from PL technology,
    and specially those related to functional programming, could be used to improve hardware design.

